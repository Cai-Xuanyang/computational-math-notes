%\documentclass[12pt,a4paper]{article}  % Use this line if this document will be released
\documentclass[12pt,a4paper,draft]{article}  % Use this line if this document is a draft
\usepackage{ifdraft}


%% Bibliography
\usepackage{etoolbox}
\newcommand{\bibfile}{\jobname.bib}  % Name of the BibTeX file.
% ref.bib should be a symbolic link to the universal BibTeX file, which should be a local copy of
% https://github.com/equipez/bibliographie/blob/main/ref.bib
% Run `getbib` in the current directory under the draft mode to get the BibTeX file containing only
% the cited references. The name will be xyz.bib if this TeX file is xyz.tex.
\newcommand{\universalbib}{ref.bib}
\ifdraft{\IfFileExists{\universalbib}{\renewcommand{\bibfile}{\universalbib}}{}}{}
% The counter `cite' is used to count the number of citations.
\newcounter{cite}
\pretocmd{\cite}{\stepcounter{cite}}{}{}


%% Add line numbers in draft mode
\RequirePackage[mathlines]{lineno}
\ifdraft{\linenumbers}{}
\renewcommand{\linenumberfont}{\normalfont\scriptsize\sffamily\color{gray}}
\setlength{\linenumbersep}{\marginparsep}


%% Geometry
%\voffset=-1.5cm \hoffset=-1.4cm \textwidth=16cm \textheight=22.0cm  % Luis' setting
\usepackage[a4paper, textwidth=16.0cm, textheight=22.0cm]{geometry}
\renewcommand{\baselinestretch}{1.2}


%% Basic packages
\usepackage{amsmath,amsthm,amssymb,amsfonts}
\usepackage{mathtools}  % Provides \coloneqq
\usepackage{empheq}
\usepackage{xcolor}
\usepackage[bbgreekl]{mathbbol}
\DeclareSymbolFontAlphabet{\mathbbm}{bbold}
\DeclareSymbolFontAlphabet{\mathbb}{AMSb}
\usepackage{bbm}
\usepackage{upgreek}
\usepackage{accents}
\usepackage{xspace}
\usepackage{rotating}
\usepackage{multirow,booktabs}
\usepackage[en-US]{datetime2}


%% Format of the table of content
\usepackage[normalem]{ulem}
\usepackage[toc,page]{appendix}
\renewcommand{\appendixpagename}{\Large{Appendix}}
\renewcommand{\appendixname}{Appendix}
\renewcommand{\appendixtocname}{Appendix}
%\usepackage{sectsty}
\setcounter{tocdepth}{2}


%% Section title style
\usepackage{sectsty}
\sectionfont{\large}
\subsectionfont{\large}


%% Some colors
\definecolor{darkblue}{rgb}{0,0.1,0.5}
\definecolor{darkgreen}{rgb}{0,0.5,0.1}
\definecolor{darkyellow}{rgb}{0.65,0.65,0.01}


%% Todo notes
\ifdraft{
	\setlength{\marginparwidth}{2.42cm}
	\usepackage[tickmarkheight=3pt,textsize=small,backgroundcolor=blue!16,linecolor=purple,bordercolor=purple]{todonotes}
}{
	\newcommand{\todo}[1]{}
	\newcommand{\listoftodos}{}
}


%% Graph, tikz and pgf
%\usepackage{subfigure}
\setlength{\unitlength}{1mm}
% The \unitlength command is a Length command. It defines the units used in the Picture Environment.
\usepackage{graphicx}
%\usepackage{tikz,tikzscale,pgf,pgfarrows,pgfnodes,filecontents,tikz-cd}
\usepackage{tikz,tikzscale,pgf}
\usetikzlibrary{arrows,arrows.meta,patterns,positioning,decorations.markings,shapes}
\usepackage{pgfplots}
\usepackage{pgfplotstable}
\usepackage[justification=centering]{caption}
\usepgfplotslibrary{fillbetween}
\pgfplotsset{compat=1.11}


%% Turn off some unharmful warnings in draft mode
%% N.B.: DO NOT use `silence` together with `hyperref`. They will cause an infinite loop.
\ifdraft{
	\usepackage{silence}
	\WarningFilter{xcolor}{Incompatible color definition on}
	\WarningFilter{hyperref}{Draft mode on}
	\WarningFilter{refcheck}{Unused label}
	\WarningFilter{microtype}{`draft' option active}
	\WarningFilter{latex}{Writing or overwriting file} % Mute the warning about 'writing/overwriting file'
	\WarningFilter{latex}{Writing file} % Mute the warning about 'writing/overwriting file'
	\WarningFilter{latex}{Tab has} % Mute the warning about 'Tab has been converted to Blank Space'
	\WarningFilter{latex}{Marginpar on page} % Mute the warning about 'Marginpar on page xx moved'
	\WarningFilter{latex}{author given} % Mute the warning about 'No \author given'
}{}


%% Hyperref, url, and email
%% N.B.: DO NOT use `silence` together with `hyperref`. They will cause an infinite loop.
\ifdraft{\usepackage{refcheck}\newcommand{\url}{\texttt}}{
	\usepackage{hyperref}
	\hypersetup{colorlinks, linkcolor=darkblue, anchorcolor=darkblue, citecolor=darkblue, urlcolor=darkblue}
	\usepackage{url}
} % Check unused labels
\newcommand{\email}{\texttt}


%% Enumerate and itemize
\usepackage{eqlist}
\usepackage{enumitem}
\setlist[itemize]{leftmargin=*}
\setlist[enumerate]{leftmargin=*,label=\normalfont{(\alph*)}}


%% Algorithm environment
\usepackage[section]{algorithm}
\usepackage{algpseudocode,algorithmicx}
\newcommand{\INPUT}{\textbf{Input}}
\newcommand{\FOR}{\textbf{For}~}
\algrenewcommand\algorithmicrequire{\textbf{Input:}}
\algrenewcommand\algorithmicensure{\textbf{Output:}}
\algrenewcommand\alglinenumber[1]{\normalsize #1.}
\newcommand*\Let[2]{\State #1 $=$ #2}


%% Theorem-like environments
\newtheorem{theorem}{Theorem}[section]
\newtheorem{conjecture}{Conjecture}[section]
\newtheorem{corollary}{Corollary}[section]
\newtheorem{exercise}{Exercise}[section]
\newtheorem{lemma}{Lemma}[section]
\newtheorem{problem}{Problem}[section]
\newtheorem{proposition}{Proposition}[section]
\newtheorem{assumption}{Assumption}[section]
\newtheorem{example}{Example}[section]
\newtheorem{question}{Question}[section]
% Change theoremstyle to ``definition'', which uses textnormal for the text.
\theoremstyle{definition}
\newtheorem{definition}{Definition}[section]
\newtheorem{remark}{Remark}[section]
% proof
\usepackage{xpatch}
\xpatchcmd{\proof}{\itshape}{\normalfont\proofnamefont}{}{}
\newcommand{\proofnamefont}{\bfseries}

%% Equation numbering
\numberwithin{equation}{section}


%% Fine tuning
\usepackage{microtype}
\usepackage[nobottomtitles*]{titlesec} % No section title at the bottom of pages
% Prevent footnote from running to the next page
\interfootnotelinepenalty=10000
% No line break in inline math
\interdisplaylinepenalty=10000
\relpenalty=10000
\binoppenalty=10000
% No widow or orphan lines
\clubpenalty=10000
\widowpenalty=10000
\displaywidowpenalty=10000


% Use @ to put 1 math unit (mu) in math
% See https://nhigham.com/2013/01/07/fine-tuning-spacing-in-latex-equations/
% and also TeXbook p. 155.
\mathcode`@="8000{\catcode`\@=\active\gdef@{\mkern1mu}}


%% Operators, commands
\usepackage{relsize}
\usepackage{nccmath}
%\DeclareMathOperator*{\mcap}{\,\medmath{\bigcap}\,}
%\DeclareMathOperator*{\mcup}{\,\medmath{\bigcup}\,}
\DeclareMathOperator*{\mcap}{\,\mathsmaller{\bigcap}\,}
\DeclareMathOperator*{\mcup}{\,\mathsmaller{\bigcup}\,}
%\renewcommand{\cap}{\mcap}
%\renewcommand{\cup}{\mcup}

\newcommand{\ceil}[1]{ {\lceil{#1}\rceil} }
\newcommand{\floor}[1]{ {\lfloor{#1}\rfloor} }

\DeclareMathOperator{\tr}{tr}
\DeclareMathOperator{\sort}{sort}
\DeclareMathOperator*{\Argmax}{Argmax}
\DeclareMathOperator*{\Argmin}{Argmin}
\DeclareMathOperator*{\Arglocmin}{Arglocmin}
\DeclareMathOperator*{\argmax}{argmax}
\DeclareMathOperator*{\argmin}{argmin}
\DeclareMathOperator*{\diag}{diag}
\DeclareMathOperator*{\Diag}{Diag}
\DeclareMathOperator{\Span}{span}
\DeclareMathOperator{\med}{med}
\DeclareMathOperator{\essinf}{essinf}
\DeclareMathOperator{\cl}{cl}
\DeclareMathOperator{\vol}{vol}
\DeclareMathOperator{\comp}{C}
\DeclareMathOperator{\sign}{sign}
\DeclareMathOperator{\rank}{rank}
\DeclareMathOperator{\range}{range}
\DeclareMathOperator{\card}{card}
\DeclareMathOperator{\diam}{diam}
\DeclareMathOperator{\dist}{dist}
\newcommand{\disth}{{\operatorname{\updelta_{\sss{H}}}}}
\newcommand{\ind}{\mathbbm{1}}
%\newcommand*{\defeq}{\stackrel{\mbox{\normalfont\tiny{\textnormal{def}}}}{=}}
\newcommand\defeq{\mathrel{\overset{\makebox[0pt]{\mbox{\normalfont\tiny\sffamily def}}}{=}}}

\newcommand{\RR}{\mathbb{R}}
\newcommand{\BB}{\mathcal{B}}
\renewcommand{\SS}{\mathbb{S}}
\newcommand{\TT}{\mathcal{T}}
\newcommand{\ZZ}{\mathbb{Z}}
\newcommand{\NN}{\mathbb{N}}
\newcommand{\FF}{\mathcal{F}}
\newcommand{\CC}{\mathbb{C}}
\newcommand{\XX}{\mathcal{X}}
\newcommand{\sset}{\mathcal{S}}
\newcommand{\pen}{h}
\newcommand{\penpar}{\mu}
\newcommand{\res}{\rho}
\newcommand{\col}{r}
\newcommand{\ofd}{\mathcal{F}}
\newcommand{\stf}[1]{\mathbb{S}^{#1}}
\newcommand{\sss}[1]{{\scriptscriptstyle{#1}}}
\newcommand{\sK}{{\scriptscriptstyle{K}}}
\newcommand{\sT}{{\scriptscriptstyle{T}}}
\newcommand{\fro}{{\scriptstyle{\textnormal{F}}}}
\newcommand{\trs}{{\scriptstyle{\mathsf{T}}}}
\newcommand{\hmt}{{\scriptstyle{{\mathsf{H}}}}}
\newcommand{\pin}{{\scriptstyle{{\mathsf{+}}}}}
\newcommand{\inv}{{-1}}
\newcommand{\adj}{*}
\newcommand{\ones}{\mathbf{1}}

\newcommand{\cs}{\text{c}}
\newcommand{\hp}{\circ}
\newcommand{\cc}{\sss{\textnormal{C}}}
\newcommand{\dec}{\sss{\textnormal{D}}}
\newcommand{\cauchy}{\sss{\textnormal{C}}}
\newcommand{\scauchy}{\sss{\textnormal{S}}}
\newcommand{\crit}{\textnormal{crit}}
\newcommand{\rsg}{\hat{\partial}}
\newcommand{\gsg}{\partial}
\newcommand{\dom}{\textnormal{dom}}
\newcommand{\tf}{{\textnormal{f}}}
\newcommand{\tg}{{\textnormal{g}}}
\newcommand{\ts}{{\textnormal{s}}}
\newcommand{\st}{\textnormal{s.t.}}
\newcommand{\etc}{{etc.}\xspace}
\newcommand{\ie}{{i.e.}\xspace}
\newcommand{\eg}{{e.g.}\xspace}
\newcommand{\etal}{{et al.}\xspace}
\newcommand{\iid}{\text{i.i.d.}\xspace}
\newcommand{\as}{\text{a.s.}\xspace}

\newcommand{\me}{\mathrm{e}}
\newcommand{\md}{\mathrm{d}}
\newcommand{\mi}{\mathrm{i}}
\newcommand{\lev}{\mathrm{lev}}
\newcommand{\bA}{\mathbf{A}}
\newcommand{\bx}{\mathbf{u}}
%\newcommand{\bb}{\mathbf{f}}
\newcommand{\bb}{\mathbf{r}}
\newcommand{\nov}{n_{\textnormal{o}}}
\xspaceaddexceptions{]\}}
% tex.stackexchange.com/questions/15252/why-does-xspace-behave-differently-for-parenthesis-vs-braces-brackets
\newcommand{\MATLAB}{\textsc{Matlab}\xspace}
\newcommand{\octave}{\mbox{GNU Octave}\xspace}
\newcommand{\prblm}{\texttt}
\DeclareMathAlphabet{\mathsfit}{T1}{\sfdefault}{\mddefault}{\sldefault}
\SetMathAlphabet{\mathsfit}{bold}{T1}{\sfdefault}{\bfdefault}{\sldefault}
\newcommand{\prbb}{\mathsfit{p}}
\newcommand{\pp}{\mathsf{p}}
\newcommand{\qq}{\mathsf{q}}
\newcommand{\ttt}{\mathsfit{t}}
\newcommand{\tol}{\varepsilon}
\newcommand{\bt}{\mathbf{t}}
\newcommand{\br}{\mathbf{r}}
\newcommand{\dd}{\mathbf{d}}
\newcommand{\ii}{\mathbf{i}}
\newcommand{\jj}{\mathbf{j}}
\newcommand{\xx}{\mathbf{x}}
\renewcommand{\pp}{\mathbf{p}}
\renewcommand{\ggg}{\mathbf{g}}
\newcommand{\GG}{\mathbf{G}}
\DeclareMathOperator{\expc}{\mathbb{E}}
\renewcommand{\Pr}{\mathbb{P}}
\newcommand{\lb}{\underline}
\newcommand{\ub}{\overline}

% mathlcal font
\DeclareFontFamily{U}{dutchcal}{\skewchar\font=45 }
\DeclareFontShape{U}{dutchcal}{m}{n}{<-> s*[1.0] dutchcal-r}{}
\DeclareFontShape{U}{dutchcal}{b}{n}{<-> s*[1.0] dutchcal-b}{}
\DeclareMathAlphabet{\mathlcal}{U}{dutchcal}{m}{n}
\SetMathAlphabet{\mathlcal}{bold}{U}{dutchcal}{b}{n}

% mathscr font (supporting lowercase letters)
%\usepackage[scr=dutchcal]{mathalfa}
%\usepackage[scr=esstix]{mathalfa}
%\usepackage[scr=boondox]{mathalfa}
%\usepackage[scr=boondoxo]{mathalfa}
\usepackage[scr=boondoxupr]{mathalfa}
%\newcommand{\model}{\mathscr{h}}
\newcommand{\model}{\tilde{f}}
\newcommand{\rmod}{F}

\newcommand{\Set}[1]{\mathcal{#1}}
\DeclareMathAlphabet{\mathpzc}{OT1}{pzc}{m}{it} % The mathpzc font
\newcommand{\slv}{\mathpzc}
% mathpzc looks great, but it stops working on 19 Feb 2020 for no reason.
%\newcommand{\slv}{\mathscr}
\newcommand{\software}{\texttt}
\DeclareMathOperator{\eff}{\mathsf{e}\;\!}
\DeclareMathOperator{\Eff}{\mathsf{E}\;\!}
\newcommand{\out}{{\text{out}}}


%% Commands for revision
\newcommand{\red}[1]{\textcolor{red}{#1}}
\newcommand{\blue}[1]{\textcolor{blue}{#1}}
\newcommand{\green}[1]{\textcolor{darkgreen}{#1}}
\newcommand{\TYPO}[1]{{\color{orange}{#1}}}
\newcommand{\MISTAKE}[1]{{\color{violet}{#1}}}
\newcommand{\REPHRASE}[1]{{\color{darkgreen}{#1}}}
\newcommand{\REVISE}[1]{{\color{blue}{#1}}}
\newcommand{\REVISEred}[1]{{\color{red}{#1}}}
\newcommand{\COMMENT}{\todo}  % Needs the todonotes package
%\newcommand{\COMMENT}[1]{\textcolor{brown}{{\small{(comment: #1)}}}}  % This puts comments inline

% Use the following if revision is finished
%\newcommand{\TYPO}{}
%\newcommand{\MISTAKE}{}
%\newcommand{\REPHRASE}{}
%\newcommand{\REVISE}{}
%\newcommand{\REVISEred}{}
%\newcommand{\COMMENT}[1]{}  % Input ignored.

% 添加 tcolorbox 支持你的自定义定理环境
\usepackage{tcolorbox}
\tcbuselibrary{theorems,breakable}
\newenvironment{badtheorem}
{\begin{tcolorbox}[colback=blue!5, colframe=blue!75!black, title=Theorem~\thetheorem~(bad)]}
	{\end{tcolorbox}}
\newenvironment{goodtheorem}
{\begin{tcolorbox}[colback=blue!5, colframe=blue!75!black, title=Theorem~\thetheorem~(good)]}
	{\end{tcolorbox}%
	\addtocounter{theorem}{-1}} 

\renewenvironment{proof}[1][\proofname]
{\begin{tcolorbox}[
		colback=blue!5,
		colframe=blue!75!black,
		title=#1,
		fonttitle=\bfseries,
		breakable
		]}
	{\end{tcolorbox}}

%%%%%%%%%%%%%%%%%%%%%%%%%%%%%%%%%%%%%%%%%%%%%%%%%%%%%%%%%%%%%%%%%%%%%%%%%%%%%%%%%%%%%%%%%%%%%%%%%%%%
\title{A Handout of Mathematical Writing}

\date{\DTMnow}

\author{Cai Xuanyang
	\thanks{}
}


\begin{document}
	
	\maketitle
	
	\begin{abstract}
		This handout is designed for learners of mathematical writing and systematically presents 
		the core principles and common problems in this field.
		
		We begin by analyzing and modifying a poorly written proof to demonstrate fundamental 
		principles of mathematical writing. Next, we discuss writing style and artistry. 
		Finally, we conclude with techniques for proofreading and typesetting.   
	\end{abstract}
	
	% 蓝色目录
	{\color{blue}
		\tableofcontents
	}
	
	\newpage
	
	\section{Introduction: Importance of Standardized Writing}
	
	Mathematical writing is not a supplement to mathematical work, but an inseparable 
	aspect of mathematical work. 
	
	Excellent mathematical writing clarifies ideas, enhances understanding, and ensures 
	accurate knowledge dissemination. However, many learners face problems that may 
	increase the cost of understanding and even conceal and stifle correct ideas, such 
	as abusing symbols, messy structure, and lack of reader awareness.
	
	This handout is based on \textit{Mathematical Writing} by Donald E. Knuth, and 
	I hope that it can offer some inspiration for your mathematical writing.
	
	\section{Core Principles for Mathematical Writing}
	
	This section is divided into three parts. In this section, we will read a poorly 
	written proof, then summarize the core principles that make writing clear, and 
	modify the bad proof.
	
	\subsection{Example: A Poorly Written Proof}
	
	In this part, we will read a poorly written proof presented as follows.
	
	\begin{theorem}
		If $L(C,P) \subseteq A_n$ and $C \neq \emptyset$, then $C\subseteq A_n$ and $P\subseteq A_n$.
	\end{theorem}
	
	\begin{proof}
		Assume that $L(C,P) \subseteq A_n$. Since $C$ is always contained in $L(C,P)$, 
		we must have $C \subseteq A_n$. Therefore, we only need to verify that $P \subseteq A_n$.
		
		Suppose, for contradiction, that $P$ is not contained in $A_n$. Then there exists 
		a vector $\mathbf{b} = (b_1,\ldots,b_n) \in P$ and indices $i<j$ such that $b_i<b_j$. 
		We will show that this leads to a contradiction.
		
		Since $C$ is nonempty, it contains some element $\mathbf{c}=(c_1,\ldots,c_n)$. 
		We know that the components of this vector satisfy $c_1 \geq \cdots \geq c_n$, 
		because $C \subseteq A_n$. 
		
		Now, for any $k \geq 0$, the vector $\mathbf{c}+k \mathbf{b}$ is an element of $A_n$. 
		In particular, if we take $k=c_i-c_j+1$, we have $k \geq 1$, and 
		\[
		c_i+kb_i \geq c_j+kb_j,
		\]
		which implies
		\[
		c_i-c_j \geq k(b_j-b_i).
		\]
		
		But $c_i-c_j=k-1$, and since $b_j-b_i \geq 1$, the inequality $k(b_j - b_i) \geq k$ 
		holds. This gives us 
		\[
		k-1 \geq k,
		\]
		which is impossible. It follows that $\mathbf{b}$ must be an element of $A_n$, 
		which means that $P \subseteq A_n$.
	\end{proof}
	
	\subsection{Analysis of the Poorly Written Proof}
	
	In this part, we will analyze the proof and present the problems sentence by sentence.
	
	The original poorly written proof contained the following issues:
	
	\begin{enumerate}
		\item \textbf{Original:} $L(C,P) \subseteq A_n$
		
		\textbf{Problems:}
		\begin{itemize}
			\item Do not start a sentence with a symbol
			\item Do not use an isolated formula
		\end{itemize}
		
		\item \textbf{Original:} $C \subset L \Rightarrow C \subset A_n$
		
		\textbf{Problem:} Do not use logical symbols $\forall$, $\exists$, $\Rightarrow$; 
		replace them with corresponding words
		
		\item \textbf{Original:} $\text{Spec} \quad p \in P, p \notin A_n \Rightarrow p_i<p_j \quad \text{for} \quad i,j$
		
		\textbf{Problems:}
		\begin{itemize}
			\item Avoid spelling errors: "Spec" is a misspelling of "Suppose"
			\item Do not use undefined symbols: Symbols $p_i$, $p_j$, $i$, $j$ are used without definition
		\end{itemize}
		
		\item \textbf{Original:} $c+p \in L \subset A_n$
		
		\textbf{Problem:} Symbols in different formulas should be separated by connecting words
		
		\item \textbf{Original:} $\because c_i+p_i \geq c_j+p_j \quad \text{but} \quad c_i \geq c_j \geq 0,p_j \geq p_i \therefore (c_i-c_j) \geq (p_i-p_j)$
		
		\textbf{Problem:} Abuse of subscripts makes the sentence ugly and hard to read
		
		\item \textbf{Original:} $\text{but} \quad \exists \quad \text{a constant} \quad k \ni c+kp \notin A_n$
		
		\textbf{Problem:} Different relations should be divided into two parts: 
		The expression "$k \ni c+kp \notin A_n$" is not a single relation
		
		\item \textbf{Original:} $\text{let} \quad k=(c_i-c_j)+1 \qquad c+kp \in L \subset A_n$
		
		\textbf{Problem:} Two key steps should be divided into two lines and linked by connecting words
		
		\item \textbf{Original:} $\because c_i+kp_i \geq c_j+kp_j \Rightarrow (c_i-c_j) \geq k(p_j-p_i)$ 
		$\Rightarrow k-1 \geq k \cdot m \qquad k,m \geq 1 \qquad \text{Contradiction}$
		
		\textbf{Problem:} Do not use sentence fragments: "Contradiction" is not a complete sentence
		
		\item \textbf{Original:} $\because p \in A_n$
		
		\textbf{Problem:} Do not use a single $\because$ or use $\because$ continuously
		
		\item \textbf{Original:} $\because L(C,P) \subset A_n \Rightarrow C,P \subset A_n \quad \text{and the lemma is true.}$
		
		\textbf{Problem:} Theorem, Lemma, Algorithm, Method should be capitalized
	\end{enumerate}
	
	\subsection{Modified Version: A Standard Proof}
	
	In this part, we present an improved version after modification based on Knuth's core principles.
	
	\begin{theorem}
		If $L(C,P) \subseteq A_n$ and $C \neq \emptyset$, then $C\subseteq A_n$ and $P\subseteq A_n$.
	\end{theorem}
	
	\begin{proof}
		Assume that $L(C,P) \subseteq A_n$. Since $C$ is always contained in $L(C,P)$, 
		we must have $C \subseteq A_n$. Therefore, we only need to verify that $P \subseteq A_n$.
		
		Suppose, for contradiction, that $P$ is not contained in $A_n$. Then there exists 
		a vector $\mathbf{b} = (b_1,\ldots,b_n) \in P$ and indices $i<j$ such that $b_i<b_j$. 
		We will show that this leads to a contradiction.
		
		Since $C$ is nonempty, it contains some element $\mathbf{c}=(c_1,\ldots,c_n)$. 
		We know that the components of this vector satisfy $c_1 \geq \cdots \geq c_n$, 
		because $C \subseteq A_n$. 
		
		Now, for any $k \geq 0$, the vector $\mathbf{c}+k \mathbf{b}$ is an element of $A_n$. 
		In particular, if we take $k=c_i-c_j+1$, we have $k \geq 1$, and 
		\[
		c_i+kb_i \geq c_j+kb_j,
		\]
		which implies
		\[
		c_i-c_j \geq k(b_j-b_i).
		\]
		
		But $c_i-c_j=k-1$, and since $b_j-b_i \geq 1$, the inequality $k(b_j - b_i) \geq k$ 
		holds. This gives us 
		\[
		k-1 \geq k,
		\]
		which is impossible. It follows that $\mathbf{b}$ must be an element of $A_n$, 
		which means that $P \subseteq A_n$.
	\end{proof}
	
	The revised version demonstrates significant advantages in \textbf{academic rigor} 
	and \textbf{expository standards}. First, it employs complete sentences and 
	standardized mathematical terminology to construct a well-structured argumentative 
	framework. This approach avoids non-standard symbols and abbreviated expressions, 
	ensuring each step has clear causes and consequences.
	
	Second, the revised version adheres to fundamental principles of mathematical writing. 
	By unifying symbolic notation, clearly defining domains, and standardizing reference 
	mechanisms, it effectively avoids common ambiguities. This enhances reliability and 
	reflects professional requirements.
	
	Finally, the revised version achieves both \textbf{reproducibility} and \textbf{verifiability}. 
	Its clear exposition enables researchers to comprehend the proof's rationale and 
	accurately assess its value, facilitating academic consensus and knowledge accumulation.
	
	Through systematic writing standards, this version transforms mathematical proofs 
	from personalized symbolic records into public knowledge carriers that meet academic 
	standards, ensuring both rigor and efficiency in scholarly communication.
	
	\section{Writing Style and Artistry}
	
	\textit{``A mathematical paper should be written so that it is easy to understand, 
		and so that it is a pleasure to read.''} — Donald E. Knuth
	
	This statement elevates mathematical writing from mere clarity to genuine artistry, 
	sitting at the intersection of rigorous science and expressive arts.
	
	\subsection{The Author's Duty: Clarity and Structure}
	
	The author's duty is to make the reader's task as easy as possible. This principle 
	demands unwavering focus on the reader's experience.
	
	\subsubsection{The Art of Explanation}
	
	Knuth illustrates this with an example: he once wrote ``The bottom of this page 
	is intentionally left blank,'' forcing readers to check unnecessarily. The better 
	version: ``This page is intentionally mostly blank.'' This small change demonstrates 
	anticipating and eliminating potential reader confusion.
	
	\subsubsection{The Symphony of Structure}
	
	Good mathematical writing has clear, precise, logical thinking but also follows 
	classical three-part structure like good storytelling: beginning, middle, and end. 
	A well-written proof states what will be proved, develops the argument step by step, 
	and summarizes accomplishments, creating a satisfying narrative arc.
	
	\subsection{The Aesthetic Dimension: Pursuing Elegance}
	
	The best mathematical writing has artistry about it—not only correct, but beautiful.
	
	\subsubsection{The Aesthetics of Notation}
	
	Every symbol, word, and punctuation mark should earn its place. Knuth shows how 
	``$1 + 2 + \cdots + n = \frac{n(n + 1)}{2} = O(n^2)$'' can be misinterpreted, 
	whereas ``$1 + 2 + \cdots + n = \frac{n(n + 1)}{2}$, so the sum is $O(n^2)$'' 
	is clearer and more elegant. Artistry lies in knowing when to replace symbols with words.
	
	\subsubsection{The Elegance of a Proof}
	
	Knuth admires ``graceful'' proofs. For $1^3 + 2^3 + \cdots + n^3 = (1 + 2 + \cdots + n)^2$, 
	algebraic induction is correct but clumsy. The elegant proof observes the sum can 
	be represented by a square of squares, whose area is obviously $(1 + 2 + \cdots + n)^2$. 
	This geometric insight transforms verification into understanding.
	
	\subsection{The Craft: Writing is Rewriting}
	
	Writing is rewriting. No one gets it right the first time.
	
	\subsubsection{The Reader as Collaborator}
	
	Write as if explaining ideas to a respected colleague. Knuth sends drafts to hundreds 
	of colleagues, incorporating feedback to create ``dialogue with the reader,'' where 
	text anticipates and answers questions before they form.
	
	\subsubsection{The Test of the Ear}
	
	Read your work aloud. If it sounds awkward to your ear, it will read awkwardly 
	to your reader's eye. Knuth revised ``The height of the box is its depth plus its height'' 
	to ``The total height of the box equals its depth plus its height,'' eliminating confusion.
	
	\subsection{The Ultimate Test}
	
	The ultimate test: Does it make mathematics seem beautiful? Knuth finds this in 
	Euler's formula $e^{i\pi} + 1 = 0$, connecting five fundamental constants. This 
	exemplifies not just showing you know something, but showing you want others to know it too.
	
	\section{Revision: Proofreading and Typesetting}
	
	\textit{``The difference between a good paper and a great paper often lies in 
		careful proofreading and elegant typesetting.''} — Donald E. Knuth
	
	\subsection{The Necessity of Proofreading}
	
	No manuscript is perfect in its first draft; proofreading is essential. Knuth 
	recommends a ``three-pass system.''
	
	\subsubsection{The First Pass: Content Verification}
	
	Focus solely on mathematical content. Read backwards, theorem by theorem, to catch 
	logical errors. Verify:
	\begin{itemize}
		\item All theorems correctly stated
		\item Proofs logically sound  
		\item No missing steps in arguments
		\item All assumptions explicitly stated
	\end{itemize}
	
	\subsubsection{The Second Pass: Language and Style}
	
	Focus on writing itself. Read aloud—your ear catches what your eye misses. Check:
	\begin{itemize}
		\item Grammatical errors and awkward phrasing
		\item Consistent notation and terminology
		\item Proper transitions between ideas
		\item Elimination of unnecessary repetition
	\end{itemize}
	
	\subsubsection{The Third Pass: Typographical Details}
	
	Hawk-eyed reading for small errors:
	\begin{itemize}
		\item Spelling mistakes and typos
		\item Punctuation errors
		\item Proper citation formatting
		\item Consistent numbering of equations and theorems
	\end{itemize}
	
	\subsection{The Art of Typesetting}
	
	Good mathematical typesetting makes structure visible to the reader.
	
	\subsubsection{Spacing and Alignment}
	
	Different mathematical objects require different spacing. In $f(x, y)$, comma has 
	space after but not before; in $\{x, y, z\}$ spacing should be balanced. Displayed 
	equations should be properly centered and numbered. Alignment in multi-line equations 
	should make mathematical structure clear.
	
	Proper spacing helps parse complex expressions without conscious effort.
	
	\subsubsection{Choosing the Right Symbols}
	
	Use distinct symbols for different concepts. Avoid visually similar symbols in 
	same context ($\theta$ and $\vartheta$, $O$ and $0$). Prefer standard notation 
	over confusing alternatives. The best notation disappears, allowing mathematics 
	to speak for itself.
	
	\subsubsection{The \TeX\ Philosophy}
	
	Let computer handle routine formatting decisions. Provide fine control for mathematical 
	elegance. Ensure consistency throughout document. Make source code readable and 
	maintainable. Beautiful typesetting shows respect for both mathematics and reader.
	
	\subsection{Practical Proofreading Techniques}
	
	\subsubsection{The Fresh Eye Method}
	
	Set work aside for at least three days before final proofreading. Distance allows 
	seeing it as reader would, not as writer who knows what was intended.
	
	\subsubsection{The Multiple Reader Approach}
	
	Have at least three types of readers:
	\begin{itemize}
		\item Expert in field for technical accuracy
		\item Mathematician from different field for clarity
		\item Non-mathematician for overall readability
	\end{itemize}
	
	\subsubsection{The Checklist System}
	
	Create personalized checklist of common errors:
	\begin{itemize}
		\item Check all cross-references
		\item Verify all promised results delivered
		\item Ensure every ``note that'' actually noteworthy
		\item Confirm all temporary notation properly defined and later discarded
	\end{itemize}
	
	\subsection{The Writer's Responsibility}
	
	Final responsibility for clear, correct, beautiful paper rests entirely with author. 
	Proofreading and typesetting are integral parts of writing process. Poorly proofread 
	paper with sloppy typesetting will be harder to read and less influential.
	
	We write mathematics to be read, not just to be correct. Care taken demonstrates 
	respect for reader and mathematics itself. When done well, these steps transform 
	good mathematical argument into great mathematical paper that stands test of time.
	
	\section{Conclusion: How to Write an Elegant Work}
	
	Mathematical writing is both art and science. Through this handout, we have learned 
	that developing writing incorporates three dimensions: \textbf{Understanding Core Principles}, 
	\textbf{Using Unique and Elegant Writing Style}, and \textbf{Continuous Revision}. 
	With deeper insight into these dimensions, you can improve your mathematical writing.
	
\end{document}